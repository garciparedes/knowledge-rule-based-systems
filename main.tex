\documentclass[10pt, a4paper,spanish]{article}

\usepackage[utf8]{inputenc}
\usepackage[spanish]{babel}

\usepackage[T1]{fontenc}

\usepackage[hmarginratio=1:1,top=32mm,columnsep=20pt]{geometry}
\usepackage[hang, small,labelfont=bf,up,textfont=it,up]{caption}

\usepackage{float}

\usepackage{graphicx}
\graphicspath{ {images/} }

\usepackage{titlesec}
\renewcommand\thesection{\Roman{section}}
\renewcommand\thesubsection{\Roman{subsection}}
\titleformat{\section}[block]{\large\scshape\centering}{\thesection.}{1em}{}
\titleformat{\subsection}[block]{\large}{\thesubsection.}{1em}{}

\usepackage{fancyhdr}
\pagestyle{fancy}
\fancyhead{}
\fancyfoot{}
\fancyhead[C]{ \today \ $\bullet$ Ingeniería del Conocimiento $\bullet$ Sistemas Basados en Reglas}
\fancyfoot[RO]{\thepage}

%-------------------------------------------------------------------------------
%	TITLE SECTION
%-------------------------------------------------------------------------------

\title{\vspace{-15mm}\fontsize{24pt}{10pt}\selectfont\textbf{Sistemas Basados en Reglas}} % Article title

\author{Sergio García Prado}
\date{\today}

%-------------------------------------------------------------------------------

\begin{document}

	\maketitle % Insert title

	\thispagestyle{fancy} % All pages have headers and footers

%-------------------------------------------------------------------------------
%	TEXT
%-------------------------------------------------------------------------------

	\begin{figure}[H]
		\begin{center}
			\includegraphics[width=0.6\textwidth]{diagnostic-assistant}
		\end{center}
	\end{figure}


	\section{Desarrollar una base de conocimiento para la versión reducida del asistente al diagnóstico que muestra la figura. Utilizar un lenguaje de tripletes O-A-V, permitiendo el uso de variables en las reglas para los Objetos y los Valores}

		\paragraph{}


	\section{Obtener la red RETE que genera el siguiente conjunto de reglas}

		\begin{figure}[H]
			\begin{center}
				\includegraphics[width=0.8\textwidth]{rete-exercise}
			\end{center}
		\end{figure}

		\paragraph{}
		La estrategia que se pretende lograr mediante la utilización de una red generada por el algoritmo RETE es generar una máquina de estados que reduzca el número de comparaciones realizadas para disparar una regla debido a la repetición de de partes del antecedente de las mismas. Por lo tanto, el diagrama de la red RETE que surge de la base de conocimiento descrita en el enunciado es la siguiente:

		\begin{figure}[H]
			\begin{center}
				\includegraphics[width=0.8\textwidth]{rete-graph}
			\end{center}
		\end{figure}

\end{document}
